\documentclass{article}
\usepackage[utf8]{inputenc}
\usepackage{amsmath}
\usepackage{hyperref}
\usepackage{lipsum}

\newcommand*\vect[1]{\bar{#1}}
\newcommand*\Ltwo[1]{\lVert #1 \rVert^{2}_{2}}

\begin{document}
\title{Machine Learning Assignment 3}
\author{Ryan Cooper}
\date{May 5, 2017}
\maketitle
The first part of the assignment asked for linear regression and ridge regression to be done using the closed form solution where \[w=(\Ltwo{X}+\lambda I)^{-1}X^{T}y\] \(X\) represents the data to regress on and \(y\) represents the data labels. \(w\) is the weight vector, and \(I\) is the identity matrix. \(\lambda\) is the hyperparameter that represents the gain on the regularization term in the loss function \[L=\Ltwo{y-Xw} + \lambda \Ltwo{w}\]
\(\lambda=0\) for linear regression. For ridge regression \(\lambda\) was chosen by finding the \(\lambda\) that gave the smallest validation error rate, and selecting \(\lambda\)s based on the following guidelines.
\begin{quote}
"You can begin by running the solver with \(\lambda\) = 400. Then, cut \(\lambda\) down by a factor of 2 and run again. Continue the process of cutting \(\lambda\) by a factor of 2 until you have models for 10 values of \(\lambda\) in total."
\end{quote}
Gradient descent was done with a learning rate of \(10^{-6}\) which was chosen by a process of trial and error. Convergence was defined as 
\begin{quote}
"the change in any coefficient between one iteration and the next is no larger than \(10^{-5}\)."
\end{quote}
Quotes were taken from the assignment description.
\par
The output of the program is given below
\begin{quote}
Linear Regression: \\
Training RMSE = 0.797519498089 \\
Testing RMSE = 0.835386861685 \\
\\
Ridge Regression:\\ 
Training RMSE = 0.803719050184\\
Testing RMSE = 0.84216331431\\
\\
Linear Regression with Gradient Descent:\\
Training RMSE = 0.819324569049\\
Testing RMSE = 0.849747523083\\
\\
Ridge Regression with Gradient Descent:\\
Training RMSE = 0.818521108773\\
Testing RMSE = 0.864392768628\\
\end{quote}
\vspace{.2cm}
Github repo: \url{https://github.com/Pillager225/Regression.git}
\end{document}